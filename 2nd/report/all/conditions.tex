\section{Система необходимых условий оптимальности}

Рассмотртим задачу Лагранжа в пространстве $\Omega = C^1(\Delta, \R^2)\times C(\Delta, \R)\times \R^2$:
\[
u(t) \in \R, \qquad
\ol{x}^T = (x,y) \in \R^2,\qquad 
\dot{\ol{x}}^T - (y,u) = 0, \qquad 
\phi(t,\ol{x}(t),u(t)) = (y,u);
\]
Далее выпишем следующий функционалы
\[
B_i(\ol{x},u,t_0,t_1) = B_i(x,y,u,0,1) =  \int\limits_0^1 f_i(t,\ol{x},u)dt + \psi_i(0,\ol{x}(0),1,\ol{x}(1)), \qquad \text{где }i = 1,\dots,4.
\]
\[
B_0 =  \int\limits_0^1 \frac{u^2}{1 + \alpha t^4} dt, \qquad f_0 = \frac{u^2}{1 + \alpha t^4}, \qquad
\psi_0 = 0;
\]
\[
B_1 =
\int\limits_0^1 xdt - 1 = 0, \qquad
f_1 = x, \qquad \psi_1 = -1;
\]
\[
B_2 = x(0), \qquad f_2 = 0, \qquad \psi_2 = x(0);
\]
\[
B_3 = y(0) - 1, \qquad f_3 = 0, \qquad \psi_3 = y(0) - 1 ;
\]
\[
B_4 = y(1), \qquad f_4 = 0, \qquad \psi_4 = y(1);
\]

Далее выпишем функцию Лагранжа
\[
\mathcal{L} = \int\limits_{0}^{1} L dt + l;
\]
где
\[
\text{лагранжиан: }
L = \sum\limits_{i=0}^{4} \lambda_i f_i(t,\ol{x},u) + \langle \ol{p}(t),\dot{\ol{x}} - \phi(t,\ol{x},u)\rangle;
\]
\[
\text{терминант: }
l = \sum\limits_{i=0}^{4} \lambda_i \psi_i(0,\ol{x}(0),1,\ol{x}(1));
\]
\[
\lambda = \left(\lambda_0, \dots, \lambda_4 \right), \qquad
\ol{p}(\cdot) = (p_x,p_y)  \in C^1(\Delta, \R^{2*})
\]
множетели лагранжа задачи, а так же функцию Понтрягина
\[
H(t,\ol{x},u,\ol{p},\lambda) = \langle \ol{p}(t), 
\phi(t,\ol{x},u)\rangle - \sum\limits_{i=0}^{4}\lambda_i f_i(t,\ol{x},u). 
\]

А теперь выпишем функции Лагранжа и Понтрягина в явном виде:
\begin{equation}
L = \lambda_0 \left(\frac{u^2}{1 + \alpha t^4}\right) +\lambda_1 x+ p_x (\dot x - y)
+ p_y (\dot y - u);
\end{equation}
\[
l = -\lambda_1 + \lambda_2 x(0) + \lambda_3(y(0) - 1) + \lambda_4 y(1);
\]
\[
H = p_x y + p_y u - \lambda_0 \left(\frac{u^2}{1 + \alpha t^4}\right) - \lambda_1 x;
\]

Далее применим к задаче оптимального управления~(\ref{formtask}) принцип максимума Понтрягина. Необходимые условия оптимальности:
\begin{enumerate}
\item Уравнения Эйлера-Лагранжа (сопряжённая система уравнений, условие стационарности по $\ol{x}$):
\begin{equation}\label{Euler}
\begin{cases}
\dot {p}_x = - \frac{\partial H}{\partial x} = \lambda_1;\\
\dot {p}_y =  - \frac{\partial H}{\partial y} = -p_x.
\end{cases}
\end{equation}
\item\label{optimal}
условие оптимальности по управлению,
\[
u = \mathrm{arg} \mathop{\rm abs}\limits_{u \in \R} \max H(u) = \mathrm{arg} \mathop{\rm abs}\limits_{u \in \R} \max \left(p_y u - \left(\frac{\lambda_0}{1+\alpha t^4}\right)  u^2\right) = \frac{p_y \left(1+\alpha t^4\right)}{2\lambda_0}
\]
при $\lambda_0 \not= 0$, так как $H(u)$ "--- парабола, с ветвями, направленными вниз (т.к. $\lambda_0 \ge 0$
"--- см.п.~\ref{notzero}), достигает максимума в вершине, при указанном значении аргумента $u$;
\item\label{trans}
условия трансверсальности по $\ol{x}$:
\[
p_x(t_k) = (-1)^k \frac{\partial l}{\partial x(t_k)}, \qquad
p_y(t_k) = (-1)^k \frac{\partial l}{\partial y(t_k)}. \qquad
\]
В нашем случае $k = 0, 1$, $t_0 = 0$, $t_1 = 1$. Значит
\[
p_x(0) = \lambda_2,
\qquad
p_x(1) = 0,
\qquad
p_y(0) = \lambda_3,
\qquad
p_y(1) = - \lambda_4. 
\] 
\item условия стационарности по $t_k$:\\ нет, так как в задаче~(\ref{formtask}) $t_k$ "--- известные константы;
\item условия дополняющей нежёсткости:\\ нет, так как в задаче~(\ref{formtask}) отсутствуют условия вида <<меньше или равно>>;
\item\label{notzero}
условие неотрицательности: $\lambda_0 \ge 0$;
\item условие нормировки (множители Лагранжа могут быть выбраны с точностью до положительного множителя);
\item\label{NERON}
НЕРОН (множители Лагранжа НЕ Равны Одновременно Нулю).
\end{enumerate}