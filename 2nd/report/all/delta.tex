
\section{Оценка глобальной погрешности.}
Для вычисления глобальной погрешности введём множество переменных $\delta_i:$ 
\[
 \delta_0 = 0, \qquad \delta_{k+1} = \mathrm{Err}_{k} + \delta_{k} \cdot e^{\int\limits_{t_k}^{t_{k+1}} \mu(s) ds} 
\]
Интеграл в предыдущем выражении можно приблизить следующим образом
\[
\int\limits_{t_k}^{t_{k+1}} \mu(s) ds = (t_{k+1} - t_k) \cdot \mathrm{Hmax} \left(\frac{J + J^T}{2} \right),
\]
где $J$ "--- матрица Якоби исходной системы дифференциальных уравнений, а Hmax "--- функция, возвращающая максимальное собственное значение полученной матрицы. 

\[
J = 
\left(
\begin{array}{cccccc}
 0 & 1 & 0 & 0 & 0 & 0 \\
 0 & 0 & 0 & 1 + \alpha t^4 & 0 & 0\\
 0 & 0 & 0 & 0 & 0 & 1 \\
 0 & 0 & 0 & 1 & 0 & 0 \\
 1 & 0 & 0 & 0 & 0 & 0 \\
 0 & 0 & 0 & 0 & 0 & 0 
\end{array}
\right); \qquad
\mathcal{A} = \frac{J + J^T}{2} = 
\left(
\begin{array}{cccccc}
 0 & \frac{1}{2} & 0 & 0 & \frac{1}{2} & 0 \\
 \frac{1}{2} & 0 & 0 & \frac{1}{2} A & 0 & 0 \\
 0 & 0 & 0 & -\frac{1}{2} & 0 & \frac{1}{2} \\
 0 & \frac{1}{2} A & -\frac{1}{2} & 0 & 0 & 0 \\
 \frac{1}{2} & 0 & 0 & 0 & 0 & 0 \\
 0 & 0 & \frac{1}{2} & 0 & 0 & 0 \\
\end{array}
\right),
\]
где $A = \left(\alpha  t^4+1\right)$.
Собственные числа $\lambda$ матрицы $\mathcal{A}$ находятся из харрактеристического уравнения
\begin{equation}
\lambda ^6-\lambda ^4 \left( 1 -\frac{\text{A}^2 }{4}\right)+\lambda ^2 \left(\frac{\text{A}^2 }{8}+\frac{1}{4}\right)-\frac{\text{A}^2}{64} = 0
\end{equation}
Сделаем замену $ \lambda^2 = z$, тогда получим следующее уравнение
\[
z ^3- z ^2 \left( 1 -\frac{\text{A}^2 }{4}\right)+z \left(\frac{\text{A}^2 }{8}+\frac{1}{4}\right)-\frac{\text{A}^2}{64} = 0
\]

По теореме Руше если на границе круга $B(0,R)$ на комплексной плоскости $\C$ с центром в точке $0$ и радиусом $R$ имеет место неравенство для двух голоморфных функций $f$ и $g$ следующего вида
\[
  |g(z)|\big|_{|z| = R}\le |f(z)|\big|_{|z|=R},
\]
то количество нулей с учётом кратности суммы $f+g$ в круге $B(0,R)$ совпадает с количеством нулей $f(z)$ в этом же круге $B(0,R)$.

Возьмём в качестве $f(z) = z^3$ и $g(z) = - z ^2 \left( 1 -\frac{\text{A}^2 }{4}\right)+z \left(\frac{\text{A}^2 }{8}+\frac{1}{4}\right)-\frac{\text{A}^2}{64}$. 

Найдём $R>0\colon |f|\ge |g|$ при $|z| = R$.

Положим $R =max\left\{ 3\left| 1 -\frac{\text{A}^2 }{4}\right|,\sqrt{\frac{\text{3A}^2 }{8}+\frac{3}{4}},\sqrt[3]{\frac{\text{3A}^2}{64}} \right\}$. Тогда
\[
  \big|g(z)\big|\bigg|_{|z| = R}\le 
 z ^2 \left| 1 -\frac{\text{A}^2 }{4}\right|+z \left(\frac{\text{A}^2 }{8}+\frac{1}{4}\right)+\frac{\text{A}^2}{64}
  \le \frac{R^3}3 + \frac{R^3}3 + \frac{R^3}3 = R^3 .
\]



