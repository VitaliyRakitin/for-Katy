\section{Численное решение краевой задачи методом стрельбы}
Краевая задача (\ref{finaltask}) решается численно методом стрельбы. В качестве параметров пристрелки выбираются недостающие для решения задачи Коши значения при $t = 0$
\[
\beta_1 = a(0), \qquad \beta_2 = p_y(0), \qquad \beta =\{ \beta_1, \beta_2 \}.
\] 

Задав эти значения каким-либо образом и решив задачу Коши на отрезке $\Delta = [0,1]$ получим соответствующие выбранному значению $\beta$ функции $x(t)[\beta]$, $y(t)[\beta]$, $p_x(t)[\beta]$, $p_y(t)[\beta]$ и, в частности, значения $p_x(1)[\beta]$, $y(1)[\beta]$. Задача Коши для системы дифференциальных уравнений~(\ref{finaltask}) с начальными условиями в нулевой момент времени решается численно явным методом Рунге-Кутты 5-го порядка, основанным на расчётных формулах Дормана-Принса 5(6) DOPRI5 с автоматическим выбором шага (то есть с контролем относительной локальной погрешности на шаге по правилу Рунге). Для решения краевой задачи необходимо подобрать значения $\beta$ так, чтобы выполнились условия:
\[
\phi(1)[\beta] = 0,\qquad y(1)[\beta] = 0.
\]
Однако всего у нас неизвестны значения трёх переменных в начальный момент времени: $p_x$, $y$, $a$. Заметим, что из краевой задачи~(\ref{finaltask}) видно, что $\dot{p}_x = a$, а значит $p_x(t) = a(t) \cdot (t-1)$, тогда при $t = 0 \colon$ $p_x(0) = -a(0)$, но $a = \mathrm{const}$, тогда $a(t) = -p_x(0)$.
cоответственно вектор-функцией невязок будет функция 
\[
\mathcal{X}(\beta) = 
\begin{pmatrix}
y(1)[\beta] \\
\phi(1)[\beta]-1
\end{pmatrix},
\]
Таким образом, в результате выбора вычислительной схемы метода стрельбы, решение краевой задачи свелось к решению системы трёх алгебраических уравнений от трёх неизвестных. Корень $\beta$ системы алгебраических уравнений $\mathcal{X}(\beta) = 0$ находится методом Ньютона с модификацией Исаева-Сонина. Решение линейной системы уравнений внутри модифицированного метода Ньютона осуществляется методом Гаусса с выбором главного элемента по столбцу, с повторным пересчётом.

Схема численного решения краевой задачи методом стрельбы выбрана таким образом, что при отсутствии ошибок в программной реализации решения задачи Коши, найденный методом Ньютона корень будет правильным (без учёта погрешности численного интегрирования), даже если внутри метода Ньютона есть какие-то ишибки. Напротив, ошибка в решении задачи Коши делает бесполезным полученный результат, даже если всё остальное запрограммировано правильно и методу Ньютона удалось найти корень.

Исходя из этого крайне важен следующий тест части программы, решающей задачу Коши, на системе дифференциальных уравнений с известным аналитическим решением.

Итерационный процесс начинается с $\beta_1^{0} = \beta_2^{0} = 0$, далее
\[
\begin{pmatrix}
\beta_1^{n+1}\\
\beta_2^{n+1}
\end{pmatrix} =
\begin{pmatrix}
\beta_1^{n}\\
\beta_2^{n}
\end{pmatrix} +
\begin{pmatrix}
\frac{\partial \mathcal{X}_1 }{\partial \beta_1^{n+1}} &
\frac{\partial \mathcal{X}_1 }{\partial \beta_2^{n+1}} \\
\frac{\partial \mathcal{X}_2 }{\partial \beta_1^{n+1}} &
\frac{\partial \mathcal{X}_2 }{\partial \beta_2^{n+1}} 
\end{pmatrix}^{-1}
\cdot
\mathcal{X}(\beta)
\]