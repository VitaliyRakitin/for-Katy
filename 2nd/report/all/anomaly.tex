\section{Анормальный случай и исследование задачи}
Исследуем возможность анормального случая $\lambda_0 = 0$. При  $\lambda_0 = 0$ из (\ref{formtask}) и (\ref{Euler}) получим систему дифференциальных уравнений:
\begin{equation}
	\begin{cases}
	\dot{x} = y;\\
	\dot{y} = u;\\
	\dot{p}_x = \lambda_1;\\
	\dot{p}_y = -p_x;
	\end{cases}
\end{equation}

Отсюда получаем, 
\[
p_x(t) = \lambda_1 t + C, \qquad p_y(t) = - \lambda_1 t - C. 
\]
Так же из условия (п.~\ref{optimal}), имеем 
\[
p_y(t) \equiv 0, \qquad
\dot{p}_y(t) \equiv 0,
\] 
иначе 
\[
u(t) = \pm \infty,
\] 
и такой управляемый процесс не является допустимым. Следовательно, 
\[
\lambda_1 t + C = 0,\qquad
\lambda_1 t = C, \qquad 
\]
где $t \in \R$, $\lambda_1, C = \mathrm{const}$, тогда
\[
\lambda_1 = C = 0, \qquad
p_x(t) \equiv 0.
\]
Из условий трансверсальности (п.~\ref{trans}) получаем 
\[
\lambda_1 = \lambda_2 = \lambda_3 = \lambda_4 = 0. 
\]
Таким образом, если $\lambda_0 = 0$, то все множители Лагранжа равны $0$ и получается противоречие с условием (п.~\ref{NERON}). Значит, анормальный случай невозможен.

Так как $\lambda_0 \not= 0$, то в силу однородности функции Лагранжа по множителям Лагранжа выберем следующее условие нормировки:
\[
\lambda_0 = \frac{1}{2},
\]
тогда из условия (п.~\ref{optimal}) определяется управление
\begin{equation}
u = p_y (1 + \alpha t^4), 
\end{equation}
%\[
%\text{где } \alpha = \{0.0;\quad 0.1; \quad 1.0; \quad 10.0\},\quad 
%t \in [0,1]. 
%\]