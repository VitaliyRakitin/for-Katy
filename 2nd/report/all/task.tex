\section{Постановка заадачи}

Рассматривается задача (номер 33) Лагранжа с фиксированным временным отрезком, без ограничений вида <<меньше или равно>>:

\begin{equation}\label{task}
\int\limits_{0}^{1} \frac{\ddot{x}^2}{1 + \alpha t^4} dt \to extr, %\qquad
\end{equation}
\[ 
\int\limits_{0}^{1} xdt = 1, \qquad x(0) = \dot{x}(1) = 0,  \qquad \dot{x}(0) = 1,\qquad
\]%\[\]
где $\alpha$ "--- известная константа, параметр задачи.

Требуется формализовать задачу как задачу оптимального управления, принципом максимума
Понтрягина свести задачу к краевой задаче, численно решить полученную краевую задачу методом стрельбы и обосновать точность полученных результатов, проверить полученные экстремали Понтрягина на оптимальность при~различных значениях параметра 
\[ \alpha = \{0.0;\quad 0.1; \quad 1.0; \quad 10.0\}.\]

\section{Формализация задачи}
Формализуем задачу для оптимального управляения. Для этого введём следующие обозначения
\[
y = \dot{x}, \qquad u = \dot{y} = \ddot{x}.
\]
где $u$ "--- управление. 

Тогда исходная система~(\ref{task}) перепишется в виде:
\begin{equation}\label{formtask}
	\begin{cases}
		\dot{x} = y;\\
		\dot{y} = u;\\
		u \in \R;\\
		\text{при $t = 0$: } x(0) = 0, \quad y(0) = 1;\\
		\text{при $t = 1$: } y(1) = 0;\\
		\int\limits_{0}^{1} xdt = 1;\\
		\int\limits_{0}^{1} \frac{u^2}{1 + \alpha t^4} dt \to \mathrm{extr}.\\
	\end{cases}
\end{equation}
